\section{Methoden}\label{methods}
In diesem Abschnitt wird die Funktionsweise von CRBMs beschrieben.
Grundlegendes Wissen über RBMs wird dabei vorausgesetzt.

\subsection{Convolutional RBM}\label{CRBM}
Die Convolutional RBM erweitert die normale RBM um die Möglichkeit kleine lokal auftretende Features in einem Bild zu erlernen.
Dieses Vorgehen entspricht einer Faltung wie sie in der Bildanalyse üblich ist.
Dabei besteht die CRBM aus mehreren kleinen RBMs, von denen jede jeweils einen Filter repräsentiert.
Die Größe der Filter wird zu Beginn definiert und anschließend jeder Filter mit zufälligen Werten initialisiert.
Diese Werte entsprechen den Weights zwischen den Neuronen der RBM und ein drei mal drei Pixel großer Filter entspricht dabei einer RBM mit neun Eingangsneuronen und einem Ausgangsneuron.
Jedes Bild wird mit jedem Filter gefaltet.
Dabei wird der Filter über das Bild pixelweise verschoben und die Filterantworten in eine Feature-Map geschrieben.
Für jedes Bild werden so viele Feature-Maps angelegt, wie es Filter gibt.
Die Filter-Maps sind anschließend so breit und hoch wie es Verschiebungen in x- und y-Richtung über das Bild gab.

\textit{Training:}
Da die Filter zufällig initialisiert wurden müssen sie schrittweise in einem Trainingsprozess verbessert werden, um sinnvolle Features im Bild zu erkennen.
Gut trainierte Filter, sollen sich möglichst voneinander unterscheiden und die Wirkung eines Kantenfilters haben.
Somit würden verschiedene Filter auch verschiedenartige Kanten in einem Bild finden.
Der Trainingsprozess selbst besteht aus drei Schritten.
Zunächst werden die Filter auf das Originalbild angewandt.
Die Filterantworten dienen dabei als Aktivierungsenergie für die Neuronen.
Diese Aktivierungsenergie wird mit Hilfe einer Logistikfunktion in eine Wahrscheinlichkeit umgerechnet, anhand derer das Neuron aktiviert oder nicht aktiviert wird.
Das Ergebnis sind die fertigen Feature-Maps.
Um das Ergebnis zu überprüfen werden Rekonstruktionen erzeugt, indem die Feature-Maps mit einem horizontal und vertikal gespiegelten Filter erneut gefaltet werden.
Für jedes Bild werden die Rekonstruktionsergebnisse der einzelnen Filter summiert.
%TODO

\textit{Max-Pooling:}
Beim Max-Pooling handelt es sich um eine Technik der Unterabtastung, die die jeweilige Feature-Map in Regionen mit einer festgelegten Gittergröße gliedert. 
Dabei wird nur der maximale Wert der aktuellen Region übernommen.
So werden die Ergebnisdaten um den Faktor der Gittergröße reduziert und somit nur die maximalen Ausschläge der gelernten Filter in der gespeichert.
Es ergibt sich dadurch ein beabsichtigter Nebeneffekt einer Translationsinvarianz korrelierend zur Gittergröße des Max-Poolings, denn durch die eingestellte Gittergröße des Max-Poolings geht die Information verloren, wo genau der bestimmte Filter innerhalb der sich ergebenden Region angeschlagen hat, da für jede Region nur ein Maximalwert errechnet wird.\newline
\begin{equation*}
\begin{matrix}
 0 & 0 & 6 & 4 \\
 0 & 1 & 7 & 3 \\
 1 & 2 & 5 & 1 \\
 0 & 0 & 3 & 0
\end{matrix} =
\begin{matrix}
 1 & 7 \\
 2 & 5  
\end{matrix}
\end{equation*}
In dem dargestellten Beispiel liegt die Regionsgröße auf $2 \cdot 2$ und so entsteht aus der $4 \cdot 4$ eine $2 \cdot 2$ Feature-Map, als Ergebnis.  


\textit{Kaskadierung:}
Um die Translationsinvarianz zu erhöhen und die Featurefilter weiter auf komplexere, signifikante und wieder auftretende Strukturen zu reduzieren, die auch als spezifische Merkmale von Objekten betrachtet werden können, können die durch das Max-Pooling sich ergebenden Feature-Maps in eine weitere CRBM gegeben werden. 
Der Vorgang der Filterung und des Max-Pooling kann so lange wiederholt werden bis die Datenflächen zu klein sind, um ein weiteres Falten oder Max-Pooling vorzunehmen.
So entsteht eine Kaskade von Einheiten, die jeweils aus einer CRBM und der anschließenden Unterabtastung bestehen.\newline
\begin{figure}[!htbp]
		\includegraphics{images/stack.jpg}
		\caption{CRBM-Kaskade \cite{Norouzi09}}
\end{figure}
