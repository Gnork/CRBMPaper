\section{Einführung}\label{introduction}
In der computergestützten Bildanalyse existieren viele Algorithmen und Herangehensweisen, um visuelle Informationen aus Bilddaten zu extrahieren.
Diese Verfahren haben gemeinsam, dass sie mit möglichst geringem Aufwand eine gute Repräsentation der Daten generieren wollen.
In klassischen Ansätzen werden dazu starr definierte Algorithmen verwendet, die meist auf die Lösung sehr spezieller Probleme abzielen.
Ein alternativer Ansatz ist die Verwendung künstlicher neuronaler Netze, die vor allem seit der Einführung der Restricted Boltzmann Machines (RBM) \cite{Hinton06} in der Bildanalyse immer größere Beachtung finden.
Künstliche neuronale Netze sind ein mathematisches Modell, das der Funktionsweise des menschlichen Gehirns nachempfunden wurde.
Die Funktion zur Beschreibung der Bilder wird dem Netz mit Hilfe von Beispieldaten antrainiert.
Die RBMs haben dieses Modell weitestgehend simplifiziert, um es nachvollziehbar und vor allem stabil zu machen.
Ein Problem, das beim Trainieren der RBMs besteht ist die fehlende Translationsinvarianz.
Dies bedeutet, dass Inhalte nicht unabhängig von ihrer Pixelposition im Bild erkannt werden.
Um diesem Problem in gewissem Maße entgegen zu wirken, wurde eine Abwandlung, die Convolutional RBM (CRBM) \cite{Norouzi09}, \cite{Lee09} eingeführt.
Diese Arbeit beschäftigt sich mit der Implementierung einer CRBM nach Norouzi \cite{NorouziMaster} in Java.