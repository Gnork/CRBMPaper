\begin{abstract}
Im Bereich des maschinellen Sehens sind diverse algorithmische Verfahren bekannt, die sich mit dem Erkennen von Bildinhalten beschäftigen.
Die Anforderungen an diese Algorithmen sind sehr hoch, weshalb es bis heute keine automatische Bilderkennung gibt, die sich annähernd mit der Wahrnehmung eines Menschen vergleichen lässt.
In einem Projekt im Rahmen des Kurses Computer Vision wird ein weiteres Verfahren vorgestellt und untersucht, das auf einem lernenden Algorithmus basiert, der unter dem Namen Convolutional Restricted Boltzmann Machine bekannt ist.
Es handelt sich dabei um ein neuronales Netz, das lernt signifikante Kanten in Bildern zu erkennen und deren Anordnungen in  translationsinvariante Featurevektoren zu reduzieren, die die entsprechenden Bilder klassifizieren sollen.
Des Weiteren ist eine Verkettung der Convolutional RBMs möglich, um eine stark limitierte Datenmenge innerhalb der Features zu erhalten.
Ein primäres Ziel ist ein Clustering von Bilddaten in verschiedene Kategorien.
Mit der hier gezeigten naiven Implementierung des neuronalen Netzes kann bereits eine rudimentäre Klassifizierung erreicht werden, was sich auch in den Testergebnissen wiederspiegelt.
Da es sich um hochdimensionale Ergebnisdaten handelt, die aus einer nicht-linearen Abbildung resultieren, sind spezielle Klassifizierungsverfahren notwendig, die aber nicht Teil dieser Implementierung sind.
Trotzdem ist erkennbar, dass die Convolutional RBM ein vielversprechender Ansatz ist, eine Bilderkennung durchzuführen.
\end{abstract}