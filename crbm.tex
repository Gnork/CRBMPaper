%%%%%%%%%%%%%%%%%%%%%%%%%%%%%%%%%%%%%%%%%
% Journal Article
% LaTeX Template
% Version 1.3 (9/9/13)
%
% This template has been downloaded from:
% http://www.LaTeXTemplates.com
%
% Original author:
% Frits Wenneker (http://www.howtotex.com)
%
% License:
% CC BY-NC-SA 3.0 (http://creativecommons.org/licenses/by-nc-sa/3.0/)
%
%%%%%%%%%%%%%%%%%%%%%%%%%%%%%%%%%%%%%%%%%

%----------------------------------------------------------------------------------------
%	PACKAGES AND OTHER DOCUMENT CONFIGURATIONS
%----------------------------------------------------------------------------------------

\documentclass[twoside]{article}

\usepackage[sc]{mathpazo} % Use the Palatino font
\usepackage[T1]{fontenc} % Use 8-bit encoding that has 256 glyphs
\linespread{1.05} % Line spacing - Palatino needs more space between lines
\usepackage{microtype} % Slightly tweak font spacing for aesthetics

\usepackage[hmarginratio=1:1,top=32mm,columnsep=20pt]{geometry} % Document margins
\usepackage{multicol} % Used for the two-column layout of the document
\usepackage[hang, small,labelfont=bf,up,textfont=it,up]{caption} % Custom captions under/above floats in tables or figures
\usepackage{booktabs} % Horizontal rules in tables
\usepackage{float} % Required for tables and figures in the multi-column environment - they need to be placed in specific locations with the [H] (e.g. \begin{table}[H])
\usepackage{hyperref} % For hyperlinks in the PDF

\usepackage{lettrine} % The lettrine is the first enlarged letter at the beginning of the text
\usepackage{paralist} % Used for the compactitem environment which makes bullet points with less space between them

\usepackage{abstract} % Allows abstract customization
\renewcommand{\abstractnamefont}{\normalfont\bfseries} % Set the "Abstract" text to bold
\renewcommand{\abstracttextfont}{\normalfont\small\itshape} % Set the abstract itself to small italic text

\usepackage{titlesec} % Allows customization of titles
\renewcommand\thesection{\Roman{section}} % Roman numerals for the sections
\renewcommand\thesubsection{\Roman{subsection}} % Roman numerals for subsections
\titleformat{\section}[block]{\large\scshape\centering}{\thesection.}{1em}{} % Change the look of the section titles
\titleformat{\subsection}[block]{\large}{\thesubsection.}{1em}{} % Change the look of the section titles

\usepackage{fancyhdr} % Headers and footers
\pagestyle{fancy} % All pages have headers and footers
\fancyhead{} % Blank out the default header
\fancyfoot{} % Blank out the default footer
\fancyhead[C]{Implementierung einer CRBM $\bullet$ Februar 2014 $\bullet$ Computer Vision} % Custom header text
\fancyfoot[RO,LE]{\thepage} % Custom footer text
% deutsche packages nachträglich eingefügt
\usepackage[utf8]{inputenc}
\usepackage{german}
\usepackage{bibgerm}

\usepackage{amsmath} % math equations -- imported by Moritz
\usepackage{graphicx} % shows pictures -- imported by Moritz


\usepackage{listings} % code snippets -- imported by Moritz 
\lstset{ %
language=Java,                % choose the language of the code
basicstyle=\footnotesize,       % the size of the fonts that are used for the code
numberstyle=\footnotesize,      % the size of the fonts that are used for the line-numbers
stepnumber=1,                   % the step between two line-numbers. If it is 1 each line will 
showspaces=false,               % show spaces adding particular underscores
showstringspaces=false,         % underline spaces within strings
showtabs=false,                 % show tabs within strings adding particular underscores
tabsize=2,          % sets default tabsize to 2 spaces
captionpos=b,           % sets the caption-position to bottom
breaklines=true,        % sets automatic line breaking
breakatwhitespace=false,    % sets if automatic breaks should only happen at whitespace
escapeinside={\%*}{*)}          % if you want to add a comment within your code
}
%----------------------------------------------------------------------------------------
%	TITLE SECTION
%----------------------------------------------------------------------------------------

\title{\vspace{-15mm}\fontsize{24pt}{10pt}\selectfont\textbf{Implementierung einer Convolutional Restricted Boltzmann Machine}} % Article title

\author{
\large
\textsc{Moritz Ufer, Radek Mackowiak, Christoph Jansen}\\[2mm] % Your name
\normalsize HTW Berlin \\ % Your institution
\vspace{-5mm}
}
\date{}

%----------------------------------------------------------------------------------------

\begin{document}

\maketitle % Insert title

\thispagestyle{fancy} % All pages have headers and footers

%----------------------------------------------------------------------------------------
%	ABSTRACT
%----------------------------------------------------------------------------------------

\begin{abstract}

\end{abstract}

%----------------------------------------------------------------------------------------
%	ARTICLE CONTENTS
%----------------------------------------------------------------------------------------

\begin{multicols}{2} % Two-column layout throughout the main article text


\section{Einführung}\label{introduction}


\section{Methoden}\label{methods}
In diesem Abschnitt wird die Funktionsweise von CRBMs beschrieben.
Grundlegendes Wissen über RBMs wird dabei vorausgesetzt.

\subsection{Convolutional RBM}\label{CRBM}
Die Convolutional RBM erweitert die normale RBM um die Möglichkeit kleine lokal auftretende Features in einem Bild zu erlernen.
Dieses Vorgehen entspricht einer Faltung wie sie in der Bildanalyse üblich ist.
Dabei besteht die CRBM aus mehreren kleinen RBMs, von denen jede jeweils einen Filter repräsentiert.
Die Größe der Filter wird zu Beginn definiert und anschließend jeder Filter mit zufälligen Werten initialisiert.
Diese Werte entsprechen den Weights zwischen den Neuronen der RBM und ein drei mal drei Pixel großer Filter entspricht dabei einer RBM mit neun Eingangsneuronen und einem Ausgangsneuron.
Jedes Bild wird mit jedem Filter gefaltet.
Dabei wird der Filter über das Bild pixelweise verschoben und die Filterantworten in eine Feature-Map geschrieben.
Für jedes Bild werden so viele Feature-Maps angelegt, wie es Filter gibt.
Die Filter-Maps sind anschließend so breit und hoch wie es Verschiebungen in x- und y-Richtung über das Bild gab.

\textit{Training:}
Da die Filter zufällig initialisiert wurden müssen sie schrittweise in einem Trainingsprozess verbessert werden, um sinnvolle Features im Bild zu erkennen.
Gut trainierte Filter, sollen sich möglichst voneinander unterscheiden und die Wirkung eines Kantenfilters haben.
Somit würden verschiedene Filter auch verschiedenartige Kanten in einem Bild finden.
Der Trainingsprozess selbst besteht aus drei Schritten.
Zunächst werden die Filter auf das Originalbild angewandt.
Die Filterantworten dienen dabei als Aktivierungsenergie für die Neuronen.
Diese Aktivierungsenergie wird mit Hilfe einer Logistikfunktion in eine Wahrscheinlichkeit umgerechnet, anhand derer das Neuron aktiviert oder nicht aktiviert wird.
Das Ergebnis sind die fertigen Feature-Maps.
Um das Ergebnis zu überprüfen werden Rekonstruktionen erzeugt, indem die Feature-Maps mit einem horizontal und vertikal gespiegelten Filter erneut gefaltet werden.
Für jedes Bild werden die Rekonstruktionsergebnisse der einzelnen Filter summiert.
%TODO

\textit{Max-Pooling:}


\section{Implementierung}\label{implementation}
Die Implementierung der CRBM wurde in Java 7 realisiert.
Die folgenden Methodenbeschreibungen beziehen sich auf die angelegte Klasse namens CRBM und nachfolgend das Clustering zur Klassifizierung.
\subsection{CRBM}
Der Kern der Klasse CRBM ist die Trainingsmethode, welche die jedes Bild als eindimensionalen Datenvektor annimmt. Des weiteren werden Lernrate und die Anzahl der Durchgänge des Trainings als Epochen in der Parameterliste übergeben.
Die Anzahl der Feature-Maps $K$, die Kernel $W_k$ und deren Größe und die Seitenlänge der eingehenden Bilder werden bei der Instanziierung im Konstruktor festgelegt, dabei ist zu beachten, das nur quadratische Bilder verarbeitet werden können. Gleichzeitig werden die Kernel mittels einer Normalverteilung zufällig initialisiert.
Der Trainingsprozess lässt sich dabei in vier Teile gliedern. Der erste Schritt ist das Errechnen der $K$-Feature-Maps als Hidden-Probabilities, der $K$-positiven Gradienten und der Hidden-States.
\begin{lstlisting}
for (int k = 0; k < K; k++) {
  PH0[i][k] = convolution(V0[i], W3D[k]);
  PH0[i][k] = logistic(PH0[i][k]);
  H0[i][k] = bernoulli(PH0[i][k]);
  Grad0[i][k] = convolution(V0[i], PH0[i][k]);
}
\end{lstlisting}
Nachfolgend das Rückrechnen des jeweiligen Bildes durch die an beiden Achsen gespiegelten $K$-Kernel der entstandenen Feature-Maps. 
Dabei werden bei der Methode $add(array1, array2)$ die so entstandenen $K$ Resultate auf $v1m$ summiert.
\begin{lstlisting}
for (int k = 0; k < W1.length; k++) {
  float[] r = convolution(data[k], W1[k]);
  add(v1m, r);
}
\end{lstlisting}
Da $V1m$ durch die doppelte Faltung kleiner ist, wird der Rand des ursprünglichen Bildes hinzugefügt.
\begin{lstlisting}
V1[i] = concatenate(V0[i], V1m[i]);
\end{lstlisting}
Der letzte Schritt ist die Berechnung der negativen Probabilities und die aus der Faltung mit dem rekonstruierten Bild entstehenden negativen Gradienten. 
In diesem Schritt erfolgt mitunter auch durch die Contrastive Divergence die Aktualisierung des $k$-ten Kernels.
\begin{lstlisting}
for (int k = 0; k < K; k++) {
  PH1[i][k] = convolution(V1[i], W[k]);
  PH1[i][k] = logistic(PH1[i][k]);
  Grad1[i][k] = convolution(V1[i], PH1[i][k]);

  for(int h = 0; h < W[0].length; h++) {
    W[k][h] = W[k][h] + learningRate * (Grad0[i][k][h] - Grad1[i][k][h]);
  }
}
\end{lstlisting}

\subsection{Clustering}
Da zur Klassifizierung Untermengen des MNIST-Set benutzt werden und jede Ziffer im Dateinamen bereits gelabelt wurde, ist von Vorteil den resultierenden Featurevektor eines Bildes mit dem Ziffernname des Dateinamen zu verknüpfen.
Die jeweiligen Clusterzentren werden aus dem Durchschnitt der jeweils gelabelten Vektordimensionen ermittelt.
\begin{lstlisting}
float[] center = new float[len];
        
for(float[] v : data){
  for(int i = 0; i < v.length; ++i){
    center[i] += v[i];
  }
}
        
for(int i = 0; i < len; ++i){
  center[i] /= size;
}
\end{lstlisting}
Um die Clusterzugehörigkeit eingehender Vektoren von Bilddaten, die nicht gelabelt wurden, zu erfahren, wird der minimale euklidische Abstand als Klassifizierung verwendet.
\begin{lstlisting}
for(int i = 0; i < center.length; ++i){
  distance += (center[i] - v[i]) * (center[i] - v[i]);
}
distance = Math.sqrt(distance);
\end{lstlisting}


\section{Ergebnisse}\label{results}


\section{Diskussion}\label{discussion}
Im Verlauf der Implementierung traten immer wieder Schwierigkeiten auf, die vor allem auf die undetaillierte Beschreibung der CRBM in den entsprechenden Artikeln zurückzuführen sind. Viele der Eigenschaften einer CRBM konnten auf mehrere Weisen interpretiert werden, weshalb das Ausprobieren verschiedener Einstellungen und Verfahren notwendig war.

Die übliche Normalisierung eines Filterkernels ist für das Training der Filterkernel nicht geeignet, da sich alle Kernel zu Boxfiltern entwickelten. Dieser Umstand wurde in keiner der Quellen beschrieben.

In der Master-Thesis von Norouzi \cite{NorouziMaster} wurden ein Trainingsalgorithmus als Pseudo-Code angegeben, an dem sich diese Implementierung orientiert. In diesem Code wurde die Verwendung eines Bias vorgeschlagen, der allerdings nicht innerhalb des Trainings verändert wird. Diverse feste Biaswerte konnten gewählt werden, die aber keinerlei Zuwachs an Klassifizierungsgenauigkeit brachten, weshalb in der aktuellen Implementierung kein Bias vorhanden ist. Vielleicht können in Zukunft Verfahren für dynamische Bias-Werte implementiert werden, die sich an den Datenbestand anpassen.
Auch wird in der Thesis die Filterung mittels eines drei-dimensionalen Kernels beschrieben, welcher bei der Kaskadierung zum Einsatz kommt,  um zu verhindern, dass  sich die Anzahl der Feature-Maps vervielfacht. Dieser drei-dimensionale Kernel wirkt somit dieser Vervielfachung in einer Kaskade entgegen. Es wird jedoch nicht erwähnt, ob oder wie dieser entsprechende Kernel gelernt wird.

Da das Clustering nicht linear erfolgt, ist die Abstandsmessung des Mittelpuktes eines Clusters nicht ausreichend, um den Effekt der CRBM genauer analysieren zu können. Erst durch Methoden, wie den Support Vector Machines kann der Fehler der Klassifizierung minimiert werden.

Abschließend ist anzumerken, dass die rudimentäre Implementierung in Java das Grundkonzept der CRBM umsetzt, aber noch keine vollständig zufriedenstellenden Ergebnisse liefert. Dazu müssen in Zukunft weitere Einstellungsmöglichkeiten getestet und ein besseres Clustering-Verfahren für multidimensionale Daten umgesetzt werden.

%----------------------------------------------------------------------------------------
%	REFERENCE LIST
%----------------------------------------------------------------------------------------

\bibliography{crbm}
\bibliographystyle{gerplain}


%----------------------------------------------------------------------------------------

\end{multicols}

\end{document}
